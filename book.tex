\documentclass[oneside,11pt]{memoir}
\input{structure.tex}
\renewcommand{\contentsname}{Съдържание}

\title{Историята на другия мъдрец}
\author{Хенри Ван Дайк}

%----------------------------------------------------------------------------------------

\begin{document}

\thispagestyle{empty}
% \ThisCenterWallPaper{1.12}{?.jpg}

\begin{tikzpicture}[remember picture,overlay]
  \node [rectangle, rounded corners, fill=white, opacity=0.75, anchor=south west, minimum width=3.3cm, minimum height=5cm] (box) at (-0.5,-10) (box){};
  \node[anchor=west, color01, xshift=-1.6cm, yshift=-1.4cm, text width=3cm, font=\sffamily\bfseries\scshape\Large] at (box.north){\thetitle};
  \node[anchor=west, color01, xshift=-1.65cm, yshift=-4.5cm, text width=3.2cm, font=\sffamily\bfseries] at (box.north){\theauthor};
\end{tikzpicture}

\newpage

\begin{KeepFromToc}
  \tableofcontents
\end{KeepFromToc}

\chapter*{}

\begin{quote}
  \fontsize{8}{10}\selectfont
  Този който се стреми към рая за да спаси душата си, \\
  Може да се придържа към пътя но няма да стигне до целта си; \\
  Докато онзи който ходи в любов може и надалеч да се скита, \\
  Бог пак ще го заведе там където са благословените~\footnote{
  \begin{quote}
    Who seeks for heaven alone to save his soul, \\
    May keep the path, but will not reach the goal; \\
    While he who walks in love may wander far, \\
    Yet God will bring him where the blessed are.
  \end{quote}
  }.
\end{quote}

\newpage

\chapter*{}
\emph{
\fontsize{9}{10}\selectfont
Знаете историята за Тримата Мъдреци от Изтока и как те пътуват отдалеч, за да
предложат своите дарове до яслите-люлка във Витлеем. Но чували ли сте някога
историята за Другият Мъдрец, който също вижда изгряващата звезда и тръгва да я
следва, но не пристига със своите събратя в присъствието на младенеца Исус? За
голямото желание на този четвърти поклонник и как то беше отречено и въпреки
това изпълнено в отрицанието; за многото му скитания и душевни изпитания; за
дългия път на неговото търсене и за странния начин по който намира Онзи, когото
търси -- ще разкажа историята както съм я чул на части в Залата на Мечтите, в
двореца на Човешкото Сърце.
}

\newpage

\chapter{Знакът в небето}
\markboth{Знакът в небето}{}

В дните, когато Август Цезар беше господар на много царе и Ирод властваше в
Йерусалим, в град Екбатана, между планините на Персия, живееше един човек на име
Артабан, мидиецът. Неговата къща беше близо до най-външната от седемте стени,
които обграждаха кралска съкровищница. От покрива си можеше да гледа над
издигащите се бойници от черно и бяло и пурпурно и синьо и червено и сребристо и
златисто, до хълмът, където летният дворец на Партските императори блестеше като
бижу в корона на седем нива.

Около домът на Артабан се простираше красива градина, плетеница от цветя и
овощни дървета, напоявани от множество потоци, спускащи се от склоновете на
Планината Оронт и озвучавана от безброй птици. Но всички цветове бледнееха в
мекия и ароматен мрак на късната септемврийска нощ и всички звуци бяха заглушени
в дълбокия чар на нейната тишина, освен пляскането на водата, като глас, полу
ридаещ и полу смеещ се под сенките. Високо над дърветата блестеше слаба светлина
през покритите със завеси сводове на горната обител, където господарят на къщата
се съветваше с приятелите си.

\end{document}
