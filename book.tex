\documentclass[oneside,10pt]{memoir}
\input{structure.tex}
\renewcommand{\contentsname}{Съдържание}

\title{Историята на другият мъдрец}
\author{Хенри Ван Дайк}

%----------------------------------------------------------------------------------------

\begin{document}

\thispagestyle{empty}
% \ThisCenterWallPaper{1.12}{?.jpg}

\begin{tikzpicture}[remember picture,overlay]
  \node [rectangle, rounded corners, fill=white, opacity=0.75, anchor=south west, minimum width=3.3cm, minimum height=5cm] (box) at (-0.5,-10) (box){};
  \node[anchor=west, color01, xshift=-1.6cm, yshift=-1.4cm, text width=3.0cm, font=\sffamily\bfseries\scshape\Large] at (box.north){\thetitle};
  \node[anchor=west, color01, xshift=-1.65cm, yshift=-4.5cm, text width=3.2cm, font=\sffamily\bfseries] at (box.north){\theauthor};
\end{tikzpicture}

\newpage

\begin{KeepFromToc}
  \tableofcontents
\end{KeepFromToc}

\chapter*{}

\begin{quote}
  \fontsize{8}{10}\selectfont
  Този който се стреми към рая за да спаси душата си, \\
  Може да се придържа към пътя но няма да стигне до целта си; \\
  Докато онзи който ходи в любов може и надалеч да се скита, \\
  Бог пак ще го заведе там където са благословените\footnote{
  \begin{quote}
    Who seeks for heaven alone to save his soul, \\
    May keep the path, but will not reach the goal; \\
    While he who walks in love may wander far, \\
    Yet God will bring him where the blessed are.
  \end{quote}
  }.
\end{quote}

\newpage

\chapter*{}
\emph{
\fontsize{9}{10}\selectfont
Знаете историята за Тримата Мъдреци от Изтока и как те пътуват отдалеч, за да
предложат своите дарове до яслите-люлка във Витлеем. Но чували ли сте някога
историята за Другият Мъдрец, който също вижда изгряващата звезда и тръгва да я
следва, но не пристига със своите събратя в присъствието на младенеца Исус? За
голямото желание на този четвърти поклонник и как то беше отречено и въпреки
това изпълнено в отрицанието; за многото му скитания и душевни изпитания; за
дългия път на неговото търсене и за странния начин по който намира Онзи, когото
търси -- ще разкажа историята както съм я чул на части в Залата на Мечтите, в
двореца на Човешкото Сърце.
}

\newpage

\chapter{Знакът в небето}
\markboth{Знакът в небето}{}

В дните, когато Август Цезар беше господар на много царе и Ирод властваше в
Йерусалим, в град Екбатана, между планините на Персия, живееше един човек на име
Артабан, мидиецът. Неговата къща беше близо до най-външната от седемте стени,
които обграждаха кралска съкровищница. От покрива си можеше да гледа над
издигащите се бойници от черно и бяло и пурпурно и синьо и червено и сребристо и
златисто, до хълмът, където летният дворец на Партските императори блестеше като
бижу в корона на седем нива.

Около домът на Артабан се простираше красива градина, плетеница от цветя и
овощни дървета, напоявани от множество потоци, спускащи се от склоновете на
Планината Оронт и озвучавана от безброй птици. Но всички цветове бледнееха в
мекия и ароматен мрак на късната септемврийска нощ и всички звуци бяха заглушени
в дълбокия чар на нейната тишина, освен пляскането на водата, като глас, полу
ридаещ и полу смеещ се под сенките. Високо над дърветата блестеше слаба светлина
през покритите със завеси сводове на горната обител, където господарят на къщата
се съветваше с приятелите си.

Той стоеше до вратата, за да посрещне гостите си -- висок, тъмен мъж на около
четиресет години, с блестящи очи, поставени близо едно до друго под широкото му
чело, и строги черти, издълбани около фините му тънки устни; челото на мечтател
и устата на войник, човек с деликатни чувства, но непреклонна воля -- един от
онези които, в каквито и времена да живеят, са родени за вътрешен конфликт и
живот на търсене.

Тогата му беше от чиста бяла вълна, наметната върху туника от коприна; а бяла,
заострена шапка, с дълги ревери отстрани, лежеше върху пуснатата му черна коса.
Това беше облеклото на древното духовенство на Влъхвите\footnote{Зороастрийски
жреци от Персийската империя (``Маги'').}, наричани огнепоклонниците.

``Добре дошли!'' — каза той с тихия си приятен глас, докато един след друг
влизаха в стаята -- ``добре дошъл, Абдус; мир на вас, Родаспе и Тигран, и на теб
татко, Абгар. Всички сте добре дошли, тази къщата се озарява с радостта от
вашето присъствие.''

Бяха деветима мъже, различаващи се значително по възраст, но приличащи си по
богатите одежди от многоцветна коприна и по масивните златни яки около вратовете
си, обозначавайки ги като Партски благородници, и по златни кръгове с крила
почиващи на гърдите им, знакът на последователите на Зороастър.

Те заеха местата си около малък черен олтар в края на стаята, където гореше слаб
пламък. Артабан, застанал до него и размахвайки клонки от тънки
тамариски\footnote{``Barsom'': сноп свещена трева.} над огъня, го хранеше със
сухи борови съчки и благоуханни масла. Тогава той започна древното песнопение на
Ясна и гласовете на неговите спътници се присъединиха към красивите възпявания
на Ахура-Мазда:
\begin{center}
\fontsize{8}{10}\selectfont
\setlength{\leftskip}{1cm}
Почитаме Божественият Дух, \\
цялата мъдрост и доброта притежаващ, \\
Заобиколени от Свети Безсмъртни, \\
дарителите на награда и благодат. \\
Делата на Неговите ръце ни правят щастливи, \\
Неговата истина и Неговата сила споделяме. \\
Възхваляваме всички неща, които са чисти, \\
защото те са Неговото единствено Творение; \\
Мислите, които са истински, и думите \\
и дела, спечелили одобрение; \\
Те са подкрепени от Него, \\
и пред тях се прекланяме. \\
Чуй ни, о, Мазда! Ти живееш \\
в истина и в небесна радост; \\
Очисти ни от лъжата и пази ни \\
от злото и робуването на греха; \\
Излей светлината и радостта от Твоят живот \\
върху нашия мрак и тъга. \\
Освети нашите градини и ниви, \\
Освети нашата работа и лъкатушене; \\
Освети цялата човешка раса, \\
Вярващи и невярващи; \\
Освети ни сега през нощта, \\
Освети ни сега с Твоята мощ, \\
Пламъкът на нашата свята любов \\
и песента на нашето поклонение, приемащ.
\end{center}

Огънят се издигаше с песнопението, пулсиращо, сякаш беше направен от музикален
пламък, докато хвърли ярка светлина през целото помещение, разкривайки неговата
скромност и великолепие.

\end{document}
