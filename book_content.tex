\chapter*{}

\begin{quote}
  \fontsize{8}{10}\selectfont
  Този който се стреми към рая за да спаси душата си, \\
  Може да се придържа към пътя но няма да стигне до целта си; \\
  Докато онзи който ходи в любов може и надалеч да се скита, \\
  Бог пак ще го заведе там където са благословените\footnote{
  \vspace{-0.5cm}
  \begin{quote}
    Who seeks for heaven alone to save his soul, \\
    May keep the path, but will not reach the goal; \\
    While he who walks in love may wander far, \\
    Yet God will bring him where the blessed are.
  \end{quote}
  }.
\end{quote}

\newpage

\chapter*{}
\emph{
\fontsize{9}{10}\selectfont
Знаете историята за Тримата Мъдреци от Изтока и как те пътуват отдалеч, за да
предложат своите дарове до яслите-люлка във Витлеем. Но чували ли сте някога
историята за Другият Мъдрец, който също вижда изгряващата звезда и тръгва да я
следва, но не пристига със своите събратя в присъствието на младенеца Исус? За
голямото желание на този четвърти поклонник и как то беше отречено и въпреки
това изпълнено в отрицанието; за многото му скитания и душевни изпитания; за
дългия път на неговото търсене и за странния начин по който намира Онзи, когото
търси -- ще разкажа части от историята както съм ги чул в Залата на Мечтите, в
двореца на Човешкото Сърце.
}

\newpage

\part{Знакът в небето}
\markboth{Знакът в небето}{}

В дните, когато Август Цезар беше господар на много царе и Ирод властваше в
Йерусалим, в град Екбатана, между планините на Персия, живееше един човек на име
Артабан, мидиецът. Неговата къща беше близо до най-външната от седемте стени,
които обграждаха кралска съкровищница. От покрива си можеше да гледа над
издигащите се бойници от черно и бяло и пурпурно и синьо и червено и сребристо и
златисто, до хълмът, където летният дворец на Партските императори блестеше като
бижу в корона на седем нива.

Около домът на Артабан се простираше красива градина, плетеница от цветя и
овощни дървета, напоявани от множество потоци, спускащи се от склоновете на
планината Оронт и озвучавана от безброй птици. Но всички цветове бледнееха в
мекия и ароматен мрак на късната септемврийска нощ и всички звуци бяха заглушени
в дълбокия чар на нейната тишина, освен пляскането на водата, като глас,
полу-ридаещ и полу-смеещ се под сенките. Високо над дърветата блестеше слаба
светлина през покритите със завеси сводове на горната обител, където господарят
на къщата се съветваше с приятелите си.

Той стоеше до вратата, за да посрещне гостите си -- висок, тъмен мъж на около
четиресет години, с блестящи очи, поставени близо едно до друго под широкото му
чело, и строги черти, издълбани около фините му тънки устни; челото на мечтател
и устата на войник, човек с деликатни чувства, но непреклонна воля -- един от
онези които, в каквито и времена да живеят, са родени за вътрешен конфликт и
живот на търсене.

Тогата му беше от чиста бяла вълна, наметната върху туника от коприна; а бяла,
заострена шапка, с дълги ревери отстрани, лежеше върху пуснатата му черна коса.
Това беше облеклото на древното духовенство на Влъхвите\footnote{Зороастрийски
жреци от Персийската империя (``Маги'').}, наричани огнепоклонниците.

``Добре дошли!'' каза той с тихия си приятен глас, докато един след друг влизаха
в стаята, ``добре дошъл, Абдус; мир на вас, Родаспе и Тигран, и на теб татко,
Абгар. Всички сте добре дошли, тази къщата се озарява с радостта от вашето
присъствие.''

Бяха деветима мъже, различаващи се значително по възраст, но приличащи си по
богатите одежди от многоцветна коприна и по масивните златни яки около вратовете
си, обозначавайки ги като Партски благородници, и по златните кръгове с крила
почиващи на гърдите им, знакът на последователите на Зороастър.

Те заеха местата си около малък черен олтар в края на стаята, където гореше слаб
пламък. Артабан, застанал до него и размахвайки клонки от тънки
тамариски\footnote{Сноп свещена трева (``Barsom'').} над огъня, го хранеше със
сухи борови съчки и благоуханни масла. Тогава той започна древното песнопение на
Ясна и гласовете на неговите спътници се присъединиха към красивите възпявания
на Ахура-Мазда:
\begin{center}
\fontsize{8}{10}\selectfont
\setlength{\leftskip}{1cm}
Почитаме Божественият Дух, \\
цялата мъдрост и доброта притежаващ, \\
Заобиколени от Светите Безсмъртни, \\
дарителите на награда и благодат. \\
Делата на Неговите ръце ни правят щастливи, \\
Неговата истина и Неговата сила споделяме. \\
Възхваляваме всички неща, които са чисти, \\
защото те са Неговото единствено Творение; \\
Мислите, които са истински, и думите \\
и дела, спечелили одобрение; \\
Те са подкрепени от Него, \\
и пред тях се прекланяме. \\
Чуй ни, о, Мазда! Ти живееш \\
в истина и в небесна радост; \\
Очисти ни от лъжата и пази ни \\
от злото и робуването на греха; \\
Излей светлината и радостта от Твоят живот \\
върху нашия мрак и тъга. \\
Освети нашите градини и ниви, \\
Освети нашата работа и криволичене; \\
Освети цялата човешка раса, \\
Вярващи и невярващи; \\
Освети ни сега през нощта, \\
Освети ни сега с Твоята мощ, \\
Пламъкът на нашата свята любов \\
и песента на нашето поклонение, приемащ.
\end{center}

Огънят се издигаше с песнопението, пулсиращо, сякаш беше направен от музикален
пламък, докато хвърли ярка светлина през целото помещение, разкривайки неговата
простота и великолепие.

Подът беше положен с тъмносини плочки с бели жилки; пиластри от усукано сребро
се открояваха на фона на сините стени; Клеристория\footnote{Висока част от
стена, с прозорци над нивото на очите.} от прозорци със
закръглени сводове над които беше окачена лазурна коприна; сводестият таван беше
настилка от сапфири, като небесата в своята чистота, осеяни със сребърни звезди.
От четирите ъгъла на покрива висяха четири златни магически колела, наричани
езиците на боговете. В източния край, зад олтара, имаше два тъмночервени стълба
от порфир; над тях трегер от същия камък, върху който беше изсечена фигурата на
крилат стрелец, със стрела поставена на тетивата и обтегнат лък.

Вратата между колоните, която се отваряше към терасата на покрива, беше покрита
с тежка завеса с цвят на зрял нар, избродиран с безброй златни лъчи, изстрелващи
се нагоре от пода. Всъщност стаята беше като тиха, звездна нощ, цялата лазурна и
сребриста, сияеща на изток с розовото обещание на зората. Тя беше, както трябва
да бъде къщата на един мъж, израз на характера и духа на господарят.

Когато песента приключи, той се обърна към приятелите си и ги покани да седнат
на дивана в западния край на стаята.

``Дойдохте тази вечер,'' каза той, оглеждайки кръга, ``по моя покана, като верни
ученици на Зороастър, да подновите вашето поклонение и разпалите отново вярата
си в Бога на Чистотата, като този огън който се разгаря на олтара. Ние се
покланяме не пред огъня, а пред Този, от който това е избраният символ, защото е
най-чистото от всички сътворени неща. То ни говори за някой, който е Светлина и
Истина. Не е ли така, татко?''

``Добре казано, сине мой,'' отговори почитаният Абгар. ``Просветените никога не
са идолопоклонници. Те повдигат воала на формата и влизат в светилището на
реалността и нова светлина и истина идват при тях непрекъснато чрез старите
символи.'' ``Чуйте ме тогава, отец мой и приятелите мои,'' каза Артабан много
тихо, ``докато ви разказвам за новата светлина и истина, които дойдоха при мен
чрез най-древните от всички знаци. Заедно търсихме тайните на природата и
изучавахме лечебните свойства на водата, огъня и растенията. Чели сме също и
книгите с пророчества, в които бъдещето е мъгливо предсказано с думи, които са
трудни за разбиране. Но най-висшето от всички учения е познанието за звездите.
Да проследиш пътят им означава да разплетеш нишките на мистерията на живота от
началото до края. Ако можехме да ги следваме безпогрешно, нищо нямаше да бъде
скрито от нас. Но не е ли все още непълно познанието ни за тях? Нима няма още
много звезди отвъд нашия хоризонт -- светлини, които са познати само на
обитателите на далечната южна земя, сред дърветата с подправки на Пунт и
златните мини на Офир?''

Сред слушателите се разнесе шепот на съгласие.

``Звездите,'' каза Тигран, ``са мислите на Вечния. Те са безброй. Но мислите на
човека могат да се преброят, като годините на живота му. Мъдростта на Влъхвите е
най-великата от всички мъдрости на земята, защото познава собственото си
невежество. И това е тайната на властта. Ние караме хората винаги да търсят и да
чакат нов изгрев. Но ние самите знаем, че тъмнината е равна на светлината и че
конфликтът между тях никога няма да приключи.''

``Това не ме удовлетворява,'' отвърна Артабан, ``защото, ако чакането трябва да
продължи вечно, ако то не може да бъде възнаградено, тогава не би било мъдрост
да гледаш и чакаш. Трябва да станем като онези нови учители на гърците, които
казват, че няма истина и че единствените мъдреци са онези, които прекарват
живота си в откриване и разобличаване на лъжите, в които светът е вярвал. Но
новият изгрев със сигурност ще дойде в определеното време. Не ни ли казват
нашите собствени книги, че това ще се случи и че хората ще видят блясъка на
голяма светлина?''

``Вярно е,'' каза гласът на Абгар; ``всеки верен ученик на Зороастър познава
пророчеството на Авеста и носи словото в сърцето си. <<В онзи ден Сосиош
Победителят ще се въздигне от многобройните пророци в източната страна. Около
него ще блести могъща светлина и той ще направи живота вечен, нетленен и
безсмъртен, а мъртвите ще възкръснат.>>''

``Това е мрачно предсказание,'' каза Тигран, ``и може би никога няма да го
разберем. По-добре е да се съобразим с нещата, които са ни под носа, и да
увеличим влиянието на Влъхвите в собствената им страна, вместо да търсим някой,
който може да е непознат и на когото трябва да преотстъпим властта си.''

Останалите сякаш одобряваха тези думи. Имаше безмълвно разбиране за явно
съгласие между тях; погледите им откликваха с това смътно изражение, което
винаги се появява след като оратор изказва убеждение спотайвало се дълго в
сърцата на неговите слушатели. Но Артабан се обърна към Абгар със сияние на
лицето и каза:

``Татко, пазил съм това пророчество в тайното място на душата си. Религия без
голяма надежда би била като олтар без жив огън. И сега пламъкът гори по-ярко, и
под светлината му прочетох непознати вести, които също са доишли от извора на
Истината и говорят дори още по-ясно за издигането на Победоносния в неговия
блясък.''

Той извади от пазвата на туниката си два малки свитъка от фин лен, с послания
върху тях, и ги разгърна внимателно на коляното си.

``В годините, които са изгубени във вековете, много преди да дойдат нашите
предци в земите на Вавилон, е имало мъдреци в Халдея, от които първите от
Влъхвите са научили небесните тайни. И от тези, Валаам синът на Веор е бил един
от най-могъщите. Чуйте думите на неговото пророчество: <<Ще изгрее звезда от
Яков и ще се въздигне скиптър от Израел.>>''\footnote{Числа 24:17.}

Устните на Тигран се свиха надолу с презрение, и каза:

``Юда беше пленник при водите на Вавилон и синовете на Яков бяха в робство на
нашите царе. Племената на Израел са разпръснати по планините като изгубени овце
и от малкото останали да обитават Юдея под игото на Рим, няма да се издигне нито
звезда, нито скиптър.''

``И въпреки това,'' отговори Артабан, ``Даниил евреинът, могъщият тълкувател на
сънища, съветникът на царе, мъдрият Валтасар, беше най-почитан и любим на нашия
велик цар Кир. Неоспорим пророк и четец на Божиите мисли, Даниил се доказа пред
нашия народ. И това са думите, които той написа.'' (Артабан прочете от вторият
свитък:) ``<<И така, знай и разбери, че от излизането на заповедта да се съгради
отново Йерусалим до княза Месия ще бъдат седем седмици и шейсет и две
седмици>>\footnote{Даниил 9:25.}''.

``Но, сине мой,'' каза със съмнение Абгар, ``това са мистични числа. Кой може да
ги тълкува, или кой може да намери ключа, който ще отключи техният смисъл?''

Артабан отговори: ``Беше показано на мен и на тримата ми другари сред Влъхвите
-- Каспар, Мелхиор и Балтазар. Търсихме в древните плочи на Халдея и изчислихме
времето. Пада се тази година. Изучавахме небето и през тази пролет видяхме две
от най-големите звезди да се приближават една до друга в знака на Рибата, който
е домът на евреите. Там видяхме и нова звезда, която свети една нощ и след това
изчезна. Сега отново двете големи планети се срещат. Тази нощ е тяхното
съединение. Тримата ми братя гледат от древния храм на Седемте Сфери, в Борсипа,
във Вавилон, а аз гледам от тук. Ако звездата се покаже отново, те ще ме чакат
десет дни в храма и тогава ще тръгнем заедно към Йерусалим, за да видим и да се
поклоним пред Обещаният, който ще се роди Цар на Израел. Вярвам, че знамението
ще дойде. Приготвих се за пътуването. Продадох къщата си и имуществото си и
купих тези три бижута -- сапфир, рубин и перла -- за да ги занеса като дар за
Царя. И ви моля да дойдете с мен на поклонението, за да се радваме заедно когато
намерим принца, който е достоен да му бъде служено.''

Докато говореше, той пъхна ръка в най-вътрешната гънка на пояса си и извади три
големи скъпоценни камъка -- един син като част от нощното небе, един по-червен
от изгряващ слънчев лъч, и един чист като върха на снежна планина по здрач -- и
ги положи върху разпръснатите ленени свитъци пред себе си.

Но приятелите му гледаха със странни и чужди очи. Воал на съмнение и недоверие
се появи на лицата им, като мъгла, пълзяща от блатата, за да скрие хълмовете. Те
се спогледаха с израз на учудване и съжаление, както тези, които са слушали
невероятни истории, разказа на един дивак или предложение за невъзможно
начинание.

Най-накрая Тигран каза: ``Артабан, това е суетен сън. Идва от прекомерно
вглеждане в звездите и подхранването на възвишени идеи. Би било по-мъдро да се
инвестира време в събиране на пари за новият храм на огъня в Чала. Никой крал
никога няма да се издигне от разбитата раса на Израел и никога няма да дойде
край на вечната борба между светлината и мрака. Този който го търси, е
преследвач на сенки. Сбогом.''

И друг каза: ``Артабан, не съм запознат с тези неща и службата ми на пазител на
кралското съкровище ме обвързва тук. Това търсене не е за мен. Но ако трябва да
го следваш, успех.''

И друг каза: ``В моята къща спи нова булка, и аз не мога нито да я оставя, нито
да я вземете със себе си на такова странно пътуване. Това търсене не е за мен.
Но нека твоите стъпки са успешни, където и да отидеш. На добър час.''

И друг каза: ``Аз съм болен и негоден за трудности, но има мъж сред моите слуги,
които ще изпратя с теб когато тръгнеш, за да ми донесе известие как се
справяш.''

Но Абгар, най-възрастният и този, който обичаше Артабан най-много, остана след
като другите си тръгнаха, и каза с тъга: ``Сине мой, може би светлината на
истината е в този знак, който се е появил в небесата и тогава той със сигурност
ще води към Принца и могъщият блясък. Или може да се окаже, че е само сянка на
светлината, както каза Тигран, и тогава този, който я следва, ще има само дълго
поклонение и безрезултатно търсене. Но е по-добре да следваш дори сянката на
най-доброто, отколкото да се задоволяваш с най-лошото. А тези, които искат да
видят прекрасни неща, често трябва да са готови да пътуват сами. Твърде стар съм
за това пътуване, но сърцето ми ще бъде спътник на поклонението денем и нощем и
ще знам краят на твоето търсене. Върви си в мир.''

Така един по един те излязоха от лазурните покои със сребърни звезди, и
Артабан остана в усамотение.

Той събра скъпоценностите и ги постави обратно в пояса си. Дълго време стоя и се
взира в пламъка който мъждукаше и изгасна върху олтара. После прекоси залата,
вдигна тежката завеса и между мрачните червени колони от порфир излезе на
терасата на покрива.

Трепетната възбуда по земята, преди тя да се събуди от нощният си сън, вече беше
започнала и хладният вятър, който предвещава зората, се спускаше надолу по
високите заснежени дерета на планината Оронт. Птици, полу-събудени, пълзяха и
чуруликаха между шумящите листа и мирисът на узряло грозде идваше на кратки
повеи от асмалъците.

Далеч над източната равнина бяла мъгла се простираше като езеро. Но където
далечният връх Загрос назъбваше западния хоризонт, небето беше ясно. Юпитер и
Сатурн се търкаляха заедно като капки бледен пламък на път да се слеят в едно.

Докато Артабан ги наблюдаваше, ето, лазурна искра се роди от тъмната бездна,
оформяща се с цветисто великолепие в пурпурна сфера издигаща се спираловидно
през жълто-оранжеви лъчи до състояние на бяло сияние. Малка и безкрайно
отдалечена, но същевременно перфектна във всяка част тя пулсираше в огромния
купол, сякаш трите бижута в пазвата на Влъхвата се бяха смесили и превърнали в
живо сърце от светлина. Той наведе глава и покри челото си с ръце.

``Това е знакът,'' каза той. ``Царят идва и аз ще отида да се срещна него.''

\part{При реките на Вавилон}
\markboth{При реките на Вавилон}{}

През цялата нощ Васда, най-бързият от конете на Артабан, чака оседлана и
обуздана в своя бокс, потупвайки земята нетърпеливо и разтърсвайки юздите си,
сякаш споделяща стремежа къв целта на господаря си, въпреки че не знаеше
значението ѝ.

Преди птиците напълно да са достигнали до силното си, високо, радостно пеене на
сутрешна песен, преди бялата мъгла да започне лениво да се вдига от полята,
другият мъдрец беше на седлото и яздеше бързо по високият път, който заобикаляше
основата на планината Оронт на запад.

Колко близко, колко интимно е другарството между мъжа и неговият любим кон на
дълго пътуване. То е тихо, необятно приятелство, общуване отвъд нуждата от думи.
Те пият от едни и същи крайпътни извори, и спят под същите звезди пазители.
Заедно усещат покоряващото заклинание на падащата нощ и разгарящата се радост от
зората. Господарят споделя вечерята си с гладния си спътник и усеща меките,
влажни устни да галят дланта на ръката му, докато се затварят над залъка хляб.
Под сивата зора той се събужда в своя бивак от нежното раздвижване на топъл,
сладък дъх върху спящото си лице и гледа нагоре в очите на своя верен спътник,
готов и очакващ дневният труд. Сигурно е, освен ако не е езичник и невярващ, с
каквото и име да зове своят Бог, той ще Му благодари за тази безгласна симпатия,
за тази няма привързаност и сутрешната му молитва ще бъде за двойна благословия
-- Боже, благослови и двама ни, пази нозете ни от подхлъзване и душите ни от
гибел!

И тогава, през пронизителния сутрешен въздух, бързите копита бият тяхната буйна
музика по пътя, отмерваща пулсирането на две сърца, които се движат със същото
нетърпеливо желание -- да превъзмогнат пространствата, да премахнат
разстоянията, да постигнат целта на пътуването.

Артабан трябваше да язди наистина мъдро и добре, ако желаеше да спази уречения
час с другите Влъхви; тъй като маршрутът беше сто и педесет
парасанга\footnote{Древно-персийска мярка за разстояние (1 парасанг $\approx$
3.9 -- 4.6 км).} а петнайсет беше максимумът, който можеше да измине за един
ден. Но той познаваше силата на Васда и напредваше без тревога, изминавайки
определеното разстояние всеки ден, макар че трябваше да пътува до късно през
нощта и сутрин много преди изгрев слънце.

Мина по кафявите склонове на планината Оронт, набраздени от скалните течения
на стотици потоци.

Той прекоси равнините на Нисеите, където знаменитите стада коне, хранещи се на
широки пасища, тръскаха глави към приближаващата Васда и препускаха надалеч с
грохота на много копита, а от блатистите ливади внезапно се издигаха ята диви
птици, които се въртяха в големи кръгове с размахването на безброй блестящи
крила издавайки пронизителни писъци на изненада.

Той прекоси плодородните полета на Конкабар, където прахът от харманите
изпълваше въздуха със златиста мъгла, наполовина скривайки огромният храм на
Астарта с неговите четиристотин колони.

В Багистан, сред богатите градини, напоявани от фонтани от скалите, той погледна
нагоре към планината, издигнала огромното си грубо чело над пътя, и видя
фигурата на Цар Дарий, който тъпче падналите си врагове, и гордия списък на
неговите войни и завоевания, издълбан високо върху лицето на вечната скала.

Над много студени и пусти проходи, пълзейки болезнено през пометените от вятъра
рамене на хълмовете; надолу по много черни планински клисури, където реката
бучеше и препускаше пред него като див водач; през много усмихнати долини, с
тераси от жълт варовик, пълни с лозя и плодови дървета; през дъбовите гори на
Карин и тъмните порти на Загрос, оградени от пропасти; в древния град Чала,
където хората на Самария са били държани в плен отдавна; и отново на открито
през огромният планински проход, изсечен през обкръжаващите хълмове, където видя
образа на върховния жрец на Влъхвите, изваян на стената от скала, с вдигната
ръка, сякаш за да благослови столетия от поклонници; покрай входа на тясното
дефиле, изпълнено от край до край с овощни градини от праскови и смокини, през
които реката Гинд се спускаше пенлива, за да го посрещне; над широките оризови
полета, където есенните изпарения разстилат мъртвешките си мъгли; по течението
на реката, под трепетни сенки на топола и тамаринд, сред южните хълмове; и навън
върху равнината, където пътят минаваше прав като стрела през стърнища и
пресъхнали ливади; покрай град Ктесифон, където властваха Партските императори,
и огромния метрополис Селевкия, който Александър построи; през завихрените
наводнения на Тигр и множеството канали на Ефрат, течащи в жълто през
царевичните земи -- Артабан продължаваше напред, и накрая пристигна в нощта на
десетия ден под разрушените стени на многолюдния Вавилон.

Васда беше почти изнемощяла и той с удоволствие би влезнал в града, за да намери
почивка и освежение за себе си и за нея. Но знаеше, че остават още три часа път
до храма на Седемте Сфери и трябва да стигне до полунощ, за да завари чакащите
си другари. Така че той не спря, а продължи да язди неотклонно през
стърнищата.

Горичка от финикови палми образуваше остров от мрак в бледожълтото море.
Навлизайки в сянката, Васда намали темпото и започна да избира пътя си
по-внимателно.

Нещо в тъмното в далечината изглежда я притесняваше. Тя надуши някаква опасност
или трудност; в сърцето ѝ обаче не се таеше желание да я избегне -- а само да
бъде подготвена за нея и да я посрещне решително, както трябва да направи един
добър кон. Горичката беше близо и тиха като гробница; листата не шумяха, птиците
не пееха.

Тя опипваше деликатно стъпките пред себе си, навела глава, пърхаше от време на
време с безпокойство. Внезапно, издаде звук на тревога и ужас и застана
неподвижно, трепереща със всеки мускул, пред тъмен обект в сянката на последната
палма.

Артабан слезна. Бледата звездна светлина разкри формата на човек, който лежи от
другата страна на пътя. Неговото скромно облекло и очертанията на изтощеното му
лице показваха, че той вероятно е един от бедните еврейски изгнаници, много от
които все още живееха в околността. Пребледнялата му кожа, суха и жълта като
пергамент, носеше белега на смъртоносната треска, която опустошаваше блатните
региони през есента. Леденината на смъртта беше в тънката му ръка и когато
Артабан я пусна, тя се свлече безучастно върху неподвижните гърди.

Той се извърна с чувство на съжаление, оставяйки тялото на това странно
погребение, което Влъхвите смятат за най-подходящо -- погребението на пустинята,
от което каните и лешоядите се издигат с тъмни крила, а хищните зверове се
измъкват крадешком, оставяйки само купчина бели кости в пясъка.

Но когато се обърна, дълга, слаба, призрачна въздишка излезе от устните на мъжа.
Кафявите, костеливи пръсти се затвориха конвулсивно върху крайчеца на робата на
Влъхвата и го задържаха здраво.

Сърцето на Артабан подскочи до гърлото му, но не от страх, а от безмълвно
негодувание към натрапчивостта на това непланирано забавяне. Как би могъл да
остане тук в тъмнината да съдейства на умиращ непознат? Какви претенции можеше
да има тази незначителна частица от човешки живот към неговото състрадание или
неговият дълг? Ако се забавеше дори и за час, той едва ли щеше да стигне до
Борсипа в уреченото време. Спътниците му биха помислили, че се е отказал от
пътуването. Щяха да отидат без него. Неговото търсене щеше да се провали.

Но ако продължи сега, човекът със сигурност щеше да умре. Ако той остане,
животът може би ще се завърне. Духът му пулсираше и трептеше от неотложността на
ситуацията. Трябва ли да рискува голямата награда на своята божествена вяра
заради една единствена проява на човешка любов? Трябва ли да се отклони, макар и
за миг, от следването на звездата, за да даде чаша студена вода на беден,
загиващ Евреин?

``Боже на истината и чистотата,'' молеше се той, ``насочи ме в светия път, път
на мъдрост, който само Ти познаваш.''

После се обърна към болния. Разхлаби хватката на ръката му и го положи на малък
насип в основата на палмата. Отпусна плътните пластове на тюрбана и отвори
дрехата над хлътналите гърди. Донесе вода от един от малките канали наблизо и
навлажни челото и устата на страдащия. Смеси едно от онези прости, но ефикасни
лекарства, които винаги носеше в пояса си -- тъй като Влъхвите бяха лекари,
както и астролози -- и го изля бавно между безцветните устни. Час след час той
се грижеше за него, както може да направи само умел лечител; и най-после силата
на мъжа се върна; той седна и се огледа наоколо.

``Кой си ти?'' каза той на грубия местен диалект ``защо ме намери тук и върна
живота ми?''

``Аз съм Артабан Влъхвата от град Екбатана и съм тръгнал към Йерусалим в търсене
на този, който ще се роди Цар на Евреите, велик Принц и Избавител на всички
хора. Не смея да отлагам повече пътуването си, защото керванът, който ме чака,
може да замине без мен. Но виж, ето всичкия хляб и вино които са ми останали, а
ето и отвара от лечебни билки. Когато силата ти се възстанови, ще можеш да
намериш домовете на Евреите сред къщите на Вавилон.''

Евреинът вдигна тържествено треперещите си ръце към небето.

``Нека Богът на Авраам, Исак и Яков благослови и подпомогне пътуването на
милостивия и да го заведе в мир до желаното му убежище. Но изчакай; нямам какво
да ти дам в замяна -- само това: мога да ти кажа, къде трябва да бъде търсен
Месията. Защото нашите пророци са казали, че той трябва да се роди не в
Йерусалим, а във Витлеем на Юда. Нека Господ те заведе в безопасност до това
място, защото си се смилил над болните.''

Вече беше доста след полунощ. Артабан яздеше припряно, а Васда, възстановена от
кратката почивка, тичаше ненаситно през тихата равнина и преплува каналите на
реката. Тя даваше всичко от себе си и бягаше над земята като газела.

Но навлизайки в последният етап на пътуването, с първите слънчеви лъчи, сянката
ѝ я надбяга. А очите на Артабан, тревожно оглеждайки голямата могила на Нимрод и
храма на Седемте Сфери, не виждаха и следа от приятелите си.

Многоцветните тераси от черно, оранжево, червено, жълто, зелено, синьо и бяло,
разбити от конвулсиите на природата и разпадащи се под неспирните удари на
човешко насилие, все още блестяха като порутена дъга в утринната светлина.

Артабан бързо заобиколи хълма, слезна от Васда и се изкачи на най-високата
тераса, гледайки на запад.

Необятната пустош на блатата се простираше до хоризонта и границата на
пустинята. Водни бикове стояха до застоялите мочурища, а чакали дебнеха в
ниските храсти; но нямаше и следа от кервана на мъдреците, наблизо или надалеч.

На ръба на терасата той видя малка пирамида от счупени тухли и под тях парче
пергамент. Той го вдигна и прочете: ``Изчакахме до след полунощ и не можем да
отлагаме повече. Отиваме да намерим Царя. Последвай ни през пустинята.'' Артабан
седна на земята и покри главата си в отчаяние.

``Как мога да прекося пустинята,'' каза той, ``без храна и с изнемощял кон?
Трябва да се върна във Вавилон, да продам сапфира и да купя керван камили и
провизии за пътуването. Може никога да не настигна приятелите си. Само Бог
милостивият знае дали няма да пропусна да видя Царя, защото се забавих да покажа
милост.''

\part{В името на едно малко дете}
\markboth{В името на едно малко дете}{}

В Залата на Мечтите, където слушах историята на другия мъдрец, настъпи тишина. И
в това безмълвие видях, но много смътно, фигурата му да минава през мрачните
вълни на пустинята, високо върху гърба на камилата си, поклащаща се неотклонно
напред като кораб над вълните.

Земята на смъртта бе разгърнала жестоката си мрежа около него. Каменистата
пустош не раждаше плодове, освен шипки и тръни. Тъмни скални форми стърчаха
тук-там над повърхността, като останките на загинали чудовища. Сухи и
негостоприемни планински вериги се издигаха пред него, набраздени с безводни
канали от антични потоци, безцветни и противни като белези по лицето на
природата. Плаващи хълмове от коварни пясъци бяха натрупани като гробници по
хоризонта. Денем, лютата жега смачкваше трептящия въздух с непоносимото си
бреме; и нито едно живо същество не се помръдваше по онемялата, замираща земя,
освен малки тушканчета които препускаха през изсъхналите храсти, или гущери,
изчезващи в пукнатините на скалата. Нощем, чакалите дебнеха и лаеха в
далечината, а лъвът караше тъмните дерета да отекват с кухия си рев, докато
остър изгарящ студ заменяше треската на деня. През жега и студ Влъхвата се
движеше неотклонно напред.

Тогава видях парковете и овощните градини на Дамаск, напоени от потоците Абана и
Фарпар, с наклонените им поляни, инкрустирани с цъфтеж, и гъсталаците им от
смирна и рози. Видях също и дългия снежен хребет на Хермон, и тъмните гори от
кедри, и Йорданска долина, и сините води на Галилейското езеро, и плодородната
равнина на Есдраелон, и хълмовете на Ефрем, и платата на Юда. През всички тях
проследих фигурата на Артабан, движеща се непрекъснато напред, докато пристигна
във Витлеем. И беше третият ден, след като тримата мъдреци бяха пристигнали на
това място, намерили Мария и Йосиф, с младенеца Исус, и бяха положили своите
дарове от злато, тамян и смирна в нозете Му.

Тогава другият мъдрец се приближи, изтощен, но пълен с надежда, носейки рубина и
перлата си, за да ги предложи на краля. ``Сега, най-после,'' каза той,
``непременно ще го намеря, макар и сам, и по-късно от моите братя. Еврейският
изгнаник ми спомена, че това е мястото, за което са говорили пророците. Тук ще
видя издигането на великата светлина. Но първо трябва да попитам за посещението
на моите братя и към коя къща ги е насочила звездата и на кого са представили
почитта си.''

Улиците на селото изглеждаха пусти и Артабан се чудеше дали всички мъже са се
качили на пасищата по хълмовете, за да свалят овцете си. От отворената врата на
ниска каменна къщурка той чу нежният звук на пеещ женски глас. Той влезна и
намери млада майка да приспива бебето си. Тя му разказа за непознатите от
далечния Изток, които се появили в селото преди три дни, и как те казали, че
звезда ги е отвела до мястото, където Йосиф от Назарет живее с жена си и
новороденото ѝ дете, и как са отдали почит на детето и са му поднесли много
скъпи дарове.

``Но пътниците отново изчезнаха,'' продължи тя, ``също толкова внезапно както
бяха дошли. Уплашихме се от странността на тяхното посещение. Не можахме да го
разберем. Човекът от Назарет взе бебето и неговата майка и избяга тайно същата
нощ, и според мълвата са се запътили надалеч към Египет. Оттогава над селото има
заклинание; нещо зло е надвиснало над него. Казват, че Римските войници идват от
Йерусалим, за да изискат нов данък от нас, и мъжете са скрили стадата дълбоко
сред хълмовете, за да го избегнат.''

Артабан слушаше нейната нежна, плаха реч а детето в ръцете ѝ вдигна поглед към
лицето му и се усмихна, протягайки розовите си ръчички, за да хване крилатия
златен кръг на гърдите му. Допирът стопли сърцето му. Наподобяваше на поздрав от
любов и доверие към някой, който е пътувал дълго в самота и затруднение, борейки
се със своите съмнения и страхове и следвайки светлина, забулена в облаци.

``Не е ли възможно това дете да е обещаният принц?'' попита той вътре в себе си,
докато докосваше меката му бузка. ``Царе са се раждали и в по-ниски къщи от
тази, а любимецът на звездите може да проплаче дори в колиба. Явно не е било
угодно на Богът на мъдростта да възнагради търсенето ми толкова скоро и толкова
лесно. Този, когото търся е заминал преди да дойда; и сега трябва да последвам
Царя в Египет.''

Младата майка положи бебето в люлката и се изправи, за да съдейства на странния
гост, който съдбата беше довела в къщата ѝ. Тя постави храна пред него,
обикновена селска дажба, но предложена на драго сърце и поради това пълна със
свежест както за душата, така и за тялото. Артабан прие с благодарност; и докато
ядеше, детето потъна в щастлива дрямка и мърмореше сладко в съня си, а тихата
стая се изпълни с чудно спокойствие.

Но изведнъж по улиците на селото се чу шум от диво объркване и врява, писъци и
женски плач, звън на медни тромпети и сблъсък на мечове, и отчаян вик:
„Войниците! Войниците на Ирод! Те убиват деца ни."

Лицето на младата майка побеля от ужас. Тя притисна детето към гърдите си и
приклекна неподвижно в най-тъмния ъгъл на стаята, покривайки го с гънките на
робата си, за да не се събуди и заплаче.

Но Артабан отиде бързо и застана на прага на къщата. Широките му рамене
изпълваха цялата вратата, а върхът на бялата му шапка почти докосна преградата.

Войниците дойдоха бързащи по улицата с окървавени ръце и капещи мечове. При вида
на непознатия във внушителните му одежди те се поколебаха от изненада. Капитанът
на групата се приближи до прага, за да го избута настрана. Но Артабан не трепна.
Лицето му беше спокойно, сякаш гледаше звездите, а в очите му гореше онова
непоколебимо излъчване, пред което дори полу-опитоменият ловен леопард се свива,
а свирепата хрътка спира в скока си. Той задържа войника мълчаливо за миг и след
това каза тихо:

``Тук няма никой освен мен, ще дам този скъпоценен камък на благоразумния
капитан, който ще ме остави на мира.''

Той показа рубина, блестящ в дланта на ръката си като великолепна капка кръв.

Капитанът беше изумен от блясъка на скъпоценния камък. Зениците на очите му се
разшириха жадно и корави бръчки на алчност се образуваха около устните му. Той
протегна ръка и взе рубина.

``Продължавайте нататък!'' изрева той на хората си: ``Тук няма дете. Къщата е
празна.''

Шумът и звънтенето на оръжия продължиха по улицата, а стремглавата ярост на
преследването мина покрай тайното убежище, където се криеше треперещата майка.
Артабан се върна в къщата, обърна лицето си на изток и се помоли:

``Боже на истината, прости греха ми! Излъгах, за да спася живота на дете. А два
от моите дарове ги няма вече. Похарчих за хората това което беше предназначено
за Бог. Ще бъда ли достоен някога да видя лицето на Царя?''

Но гласът на жената, плачеща от радост в сянката зад него, каза много нежно:

``Защото ти спаси живота на моето мъниче, нека Господ те благослови и те пази;
Господ да осияе с лицето Си над тебе и да ти покаже милост; Господ да издигне
лицето Си над тебе и да ти даде мир.''\footnote{Числа 6:24-26.}

\part{По скритият път на тъгата}
\markboth{По скритият път на тъгата}{}

В Залата на Мечтите отново настъпи тишина, по-дълбока и по-загадъчна от първата,
и аз разбрах, че годините на Артабан текат много бързо под забвението на тази
вкопчваща се мъгла, и зървах само за миг тук и там, реката на живота му, сияеща
през сенките, които скриваха нейното течение.

Видях го да се движи сред тълпи от хора в многолюден Египет, търсейки навсякъде
следи от семейството, дошло от Витлеем, и ги намираше под разтворените корони на
яворите на Хелиополис и под стените на Римската крепост на Нов Вавилон край Нил
-- следи, толкова бледи и смътни, че изчезваха пред него непрекъснато, като
отпечатъците по твърдия речен пясък които блестят за миг от влагата и след това
се изгубват.

Видях го отново в подножието на пирамидите, които издигаха острите си върхове
към ослепителния жълтеникав блясък на залезното небе, вечни паметници на
нетрайната слава и неувяхващата надежда на човек. Той вдигна поглед към
огромното лице на приклекналия Сфинкс и напразно се опита да разчете спокойните
ѝ очи и усмихнати устни. Наистина ли беше подигравка с всички усилия и всички
стремежи, както беше казал Тигран -- жестоката шега на една гатанка, която няма
отговор, търсене, което никога не може да успее? Или имаше нотка на съжаление и
насърчение в тази непроницаема усмивка -- обещание, че дори победеният ще
постигне победа, разочарованият ще открие награда, невежият ще стане мъдър, и
слепият ще прогледне, а скитащият най-накрая ще пристигне в своето убежище?

Видях го отново в уединена къща в Александрия, да се съветва с Еврейски равин.
Почитаният мъж, наведен над ролките пергамент, върху които бяха написани
пророчествата на Израел, прочете на глас прочувствени думи, които предсказват
страданията на обещания Месия -- презрян и отхвърлен от хората, човек на
покаянието и познайник на скръбта.

``И запомни, сине мой,'' каза той, като впери дълбоко поставените си очи в
лицето на Артабан, ``кралят, който търсиш, няма да бъде намерен в дворец, нито
сред богатите и могъщите. Ако светлината на светът и блясъка на Израел са
определени да дойдат с великолепието на земния разкош, това сигурно вече се е
случило много отдавна. Защото никой син на Авраам никога вече няма да съперничи
на силата, която Йосиф е имал в дворците на Египет, или на великолепието на
Соломон, коронован между лъвовете в Йерусалим. Но светлината, която чака светът,
е нова светлина, славата, която ще възкръсне от търпеливо и триумфално
страдание. И царството, което ще бъде установено завинаги, е ново царство,
привилегия на съвършена и непобедима любов. Не знам нито как това ще се случи,
нито как размирните царе и народи на земята ще бъдат склонени да признаят
Месията и да му отдадат почит. Но едно знам. Тези, които Го търсят, е хубаво да
погледнат сред бедните и смирените, скърбящите и потиснатите.''

Така виждах другия мъдрец отново и отново, пътувайки от място на място търсейки
сред хората от еврейската диаспора, където малкото семейство от Витлеем може би
бе намерило убежище. Премина през страни, където глад беше налегнал земята и
бедните ридаеха за хляб. Отсядаше в поразени от чумата градове, където болните
изнемогваха в горчивата компания на безпомощна мизерия. Посети потиснатите и
страдащите в мрака на подземни затвори, и претъпканата окаяност на пазари за
роби, и изтощителния труд на галерите. Въпреки че не намери пред кого да се
преклони в целия този многолюден и сложен свят на мъка, той намери много хора,
на които да помогне. Хранеше гладните, обличаше голите, изцеляваше болните и
утешаваше осъдените; годините му минаваха по-бързо от совалката на тъкача, която
проблясва напред-назад през стана, докато платното расте и невидимият мотив е
завършен.

Сякаш почти беше забравил търсенето си. Но веднъж го видях за миг, докато стоеше
сам при изгрев слънце, чакайки пред вратите на римски затвор. Той беше извадил
от тайник в пазвата си перлата, последната от неговите скъпоценности. Докато я
гледаше, нежен блясък, мека и преливаща светлина, пълна с променящи се лазурни и
розови отблясъци, трептяха на повърхността ѝ. Сякаш бе запазила отражение от
цветовете на изгубените сапфир и рубин. Така и дълбоката, тайна цел на един
благороден живот привлича в себе си спомените за минали радости и отминала
скръб. Всичко, което ѝ е помогнало, всичко, което ѝ е попречило, се прелива в
самата ѝ същност чрез изкусна магия. Тя става все по-ярка и по-ценна, колкото
по-дълго се носи близо до топлината на биещото сърце. Тогава, унесен в мисли за
тази перла и за нейното значение, най-после чух края на историята на другия
мъдрец.

\part{Перла с висока стойност}
\markboth{Перла с висока стойност}{}

Трийсет и три години от живота на Артабан бяха отминали и той все още беше
поклонник и търсещ светлина. Косите му, някога по-тъмни от скалите на Загрос
вече бяха бели като зимния сняг който ги покриваше. Очите му, които някога
проблясваха като огнени пламъци, бяха мъждукащи като жарава тлееща сред пепелта.

Похабен, изтощен и готов да умре, но все още търсещ Царя, той бе дошъл в
Йерусалим за последен път. Често беше посещавал светия град преди и беше
претърсил всичките му алеи, претъпкани коптори и черни затвори, без да открие и
следа от семейството от Назарет, което отдавна бе избягало от Витлеем. Но сега
изглеждаше, че трябва да направи още едно усилие и нещо прошепна в сърцето му,
че най-накрая може да успее. Беше сезонът на Пасхата. Градът беше претъпкан с
чуждоземци. Децата на Израел, разпръснати в далечни земи по целия свят, се бяха
върнали в Храма за големия празник, и в продължение на много дни по тесните
улички цареше объркване от много диалекти.

Но на този ден имаше необикновена възбуда, видима в множеството. Небето беше
забулено със зловещ мрак и потоци от вълнение сякаш проблясваха през тълпата
като тръпката, която разтърсва гората в навечерието на буря. Тайно течение
помиташе всички в една посока. Тропота на сандали и тихият, плътен звук на
хиляди боси крака, влачещи се по камъните, се чуваха непрестанно по улицата,
която води към портата на Дамаск.

Артабан се присъедини към група хора от собствената си страна, Партски евреи,
които бяха дошли да празнуват Пасхата, и ги попита за причината за суматохата
и къде отиват.

``Отиваме,'' отговориха те ``на място, наречено Голгота, отвъд градските стени,
където ще има екзекуция. Не си ли чул какво се е случило? Двама известни крадци
ще бъдат разпънати на кръст, а с тях и друг, наречен Исус от Назарет, човек,
който е извършил много чудни дела сред хората, и затова те Го обичат. Но
свещениците и старейшините са казали, че той трябва да умре, защото се е
провъзгласил за Син Божий. А Пилат го изпрати на кръста, защото е казал, че е
<<Цар на Евреите>>''.

Колко неочаквано се стовариха тези познати думи върху умореното сърце на
Артабан! Те го бяха водили цял живот по суша и море. А сега идваха при него
мрачно и мистериозно като послание на отчаяние. Царят бе дошъл, но беше
отхвърлен и изгонен. Беше на път да загине. Може би вече умираше. Може ли да е
същият, който е бил роден във Витлеем преди трийсет и три години, при чието
раждане звездата се е появила на небето и за чието идване са говорили пророците?

Сърцето на Артабан биеше несигурно с това смътно опасение, което
е вълнението на старостта. Но той си каза, ``Пътищата Божи са по-странни от
мислите на хората и може би най-накрая ще намеря Царя в ръцете на Неговите
врагове, и ще бъда навреме за да предложа перлата си за Неговия откуп, преди да
умре.''

Така старецът следваше множеството с бавни и болезнени стъпки към града и
портата на Дамаск. Точно отвъд входа на караулното помещение, отряд Македонски
войници слезе по улицата, влачейки младо момиче с разкъсана рокля и разрошена
коса. Когато Влъхвата спря, за да я погледне със състрадание, тя внезапно се
откъсна от ръцете на своите мъчители и се хвърли в краката му, прегръщайки го
около коленете. Беше видяла бялата му шапка и крилатия кръг на гърдите му.

``Смили се над мен,'' извика тя, ``и ме спаси в името на Бога на Чистота! Аз
също съм дъщеря на неподправената религия, която се проповядва от Влъхвите. Баща
ми беше търговец от Партия, но той е мъртъв, а аз съм задържана заради дълговете
му и ще бъда продадена като робиня. Спаси ме от по-лошо от смъртта!''

Артабан потрепери.

Това беше старият конфликт в душата му, който го бе споходи в палмовата горичка
на Вавилон и в къщурката във Витлеем -- конфликтът между очакването на вярата и
импулса на любовта. Два пъти дарът, който беше посветил на религиозно
поклонение, беше отнет от ръцете му в служба на човечеството. Това беше третото
изпитание, последната проверка, окончателният и неотменим избор.

Дали това беше големият му шанс или последното му изкушение? Не можеше да каже.
Само едно нещо беше ясно в мрака на съзнанието му -- беше неизбежно. А
неизбежното не идва ли от Бог?

Само едно нещо беше сигурно за раздвоеното му сърце -- да спаси това безпомощно
момиче би било истински акт на любов. А не е ли любовта светлината на душа?

Той извади перлата от пазвата си. Тя никога не беше изглеждала толкова сияеща,
толкова ярка, толкова пълна с нежен, жив блясък. Той я положи в ръката на
робинята.

``Това е твоят откуп, дъще! Последната ми ценност, пазех я за Избрания.''

Докато той говореше, тъмнината на небето се сгъсти и земята се разтресе,
надигайки се конвулсивно като гърдите на човек, който се бори с огромна скръб.

Стените на къщите се клатеха. Разместени камъни падаха по улицата. Облаци прах
изпълниха въздуха. Войниците избягаха в ужас, олюлявайки се като пияни мъже. А
Артабан и момичето, което той бе откупил, бяха приклекнали безпомощно под
стената на Преториума.

%% Имах доста проблеми с този параграф (сигурен съм че може да се преведе по-добре)
От какво имаше да се страхува? За какво трябваше да живее? Беше дал и последната
останка от почетните си дарове за Царя. Беше се разделил с последната надежда да
Го намери. Търсенето бе приключило и то безуспешно. Но дори в тази мисъл, приета
и прегърната, имаше мир. Не беше примирение. Не беше подчинение. А нещо
по-дълбоко и търсещо. Знаеше, че всичко е наред, защото беше правил най-доброто,
което може, ден след ден. Той беше верен на светлината, която му беше дадена. Бе
търсил и още. И ако не я намери, ако провал е всичко което произлезе от живота
му, то без съмнение това е най-доброто, което е било възможно. Той не успя да
види откровението на ``вечния живот, нетленен и безсмъртен''. Но знаеше, че дори
и да можеше да изживее земния си живот отново, не би могло той да бъде различен.

Още една продължителна пулсация на земетресението разклати земята. Тежка
керемида, отчупена от покрива, падна и удари стареца в слепоочието. Той лежеше
блед и без дъх, сивата му глава опряна на рамото на младото момиче, със стичаща
се от раната кръв. Докато тя се навеждаше над него, страхувайки се, че той е
мъртъв, в здрача се чу глас, много скромен и тих, като музика, звучаща отдалеч,
в която нотите са ясни, но думите са изгубени. Момичето се обърна за да види
дали някой не говори от прозореца над тях, но не видя никого.

Тогава устните на стареца започнаха да се движат, сякаш в отговор, и тя го чу да
казва на партски език:

``Не е така, Господарю мой! Защото кога те видях гладен и те нахраних? Или жаден
и те напоих? Кога те видях странник и те приех? Или гол, и те облякох? Кога те
видях болен или в тъмница и дойдох при теб? Трийсет и три години те търсих; но
никога не съм виждал лицето ти, нито съм ти служил, Царю мой.''

Той спря и сладкият глас се появи отново. И отново девойката го чу, много слабо
и отдалеч. Но сега изглеждаше, че тя разбра думите:

``Истината ви казвам: доколкото сте сторили това на едного от тия Мои най-малки братя,
Мене сте го сторили.''\footnote{Матей 40.}

Спокойно излъчване на удивление и радост озари бледото лице на Артабан като
първия лъч на зората върху заснежен планински връх. Един дълъг, последен дъх на
облекчение излезе нежно от устните му.

Пътуването му приключи. Даровете му бяха приети. Другият мъдрец бе намерил Царя.
