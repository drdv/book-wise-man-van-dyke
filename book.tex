\documentclass[oneside,10pt]{memoir}
\input{structure.tex}
\renewcommand{\contentsname}{Съдържание}

\title{Историята на другият мъдрец}
\author{Хенри Ван Дайк}

%----------------------------------------------------------------------------------------

\begin{document}

\thispagestyle{empty}
% \ThisCenterWallPaper{1.12}{?.jpg}

\begin{tikzpicture}[remember picture,overlay]
  \node [rectangle, rounded corners, fill=white, opacity=0.75, anchor=south west, minimum width=3.3cm, minimum height=5cm] (box) at (-0.5,-10) (box){};
  \node[anchor=west, color01, xshift=-1.6cm, yshift=-1.4cm, text width=3.0cm, font=\sffamily\bfseries\scshape\Large] at (box.north){\thetitle};
  \node[anchor=west, color01, xshift=-1.65cm, yshift=-4.5cm, text width=3.2cm, font=\sffamily\bfseries] at (box.north){\theauthor};
\end{tikzpicture}

\newpage

\begin{KeepFromToc}
  \tableofcontents
\end{KeepFromToc}

\chapter*{}

\begin{quote}
  \fontsize{8}{10}\selectfont
  Този който се стреми към рая за да спаси душата си, \\
  Може да се придържа към пътя но няма да стигне до целта си; \\
  Докато онзи който ходи в любов може и надалеч да се скита, \\
  Бог пак ще го заведе там където са благословените\footnote{
  \begin{quote}
    Who seeks for heaven alone to save his soul, \\
    May keep the path, but will not reach the goal; \\
    While he who walks in love may wander far, \\
    Yet God will bring him where the blessed are.
  \end{quote}
  }.
\end{quote}

\newpage

\chapter*{}
\emph{
\fontsize{9}{10}\selectfont
Знаете историята за Тримата Мъдреци от Изтока и как те пътуват отдалеч, за да
предложат своите дарове до яслите-люлка във Витлеем. Но чували ли сте някога
историята за Другият Мъдрец, който също вижда изгряващата звезда и тръгва да я
следва, но не пристига със своите събратя в присъствието на младенеца Исус? За
голямото желание на този четвърти поклонник и как то беше отречено и въпреки
това изпълнено в отрицанието; за многото му скитания и душевни изпитания; за
дългия път на неговото търсене и за странния начин по който намира Онзи, когото
търси -- ще разкажа историята както съм я чул на части в Залата на Мечтите, в
двореца на Човешкото Сърце.
}

\newpage

\chapter{Знакът в небето}
\markboth{Знакът в небето}{}

В дните, когато Август Цезар беше господар на много царе и Ирод властваше в
Йерусалим, в град Екбатана, между планините на Персия, живееше един човек на име
Артабан, мидиецът. Неговата къща беше близо до най-външната от седемте стени,
които обграждаха кралска съкровищница. От покрива си можеше да гледа над
издигащите се бойници от черно и бяло и пурпурно и синьо и червено и сребристо и
златисто, до хълмът, където летният дворец на Партските императори блестеше като
бижу в корона на седем нива.

Около домът на Артабан се простираше красива градина, плетеница от цветя и
овощни дървета, напоявани от множество потоци, спускащи се от склоновете на
Планината Оронт и озвучавана от безброй птици. Но всички цветове бледнееха в
мекия и ароматен мрак на късната септемврийска нощ и всички звуци бяха заглушени
в дълбокия чар на нейната тишина, освен пляскането на водата, като глас,
полуридаещ и полусмеещ се под сенките. Високо над дърветата блестеше слаба
светлина през покритите със завеси сводове на горната обител, където господарят
на къщата се съветваше с приятелите си.

Той стоеше до вратата, за да посрещне гостите си -- висок, тъмен мъж на около
четиресет години, с блестящи очи, поставени близо едно до друго под широкото му
чело, и строги черти, издълбани около фините му тънки устни; челото на мечтател
и устата на войник, човек с деликатни чувства, но непреклонна воля -- един от
онези които, в каквито и времена да живеят, са родени за вътрешен конфликт и
живот на търсене.

Тогата му беше от чиста бяла вълна, наметната върху туника от коприна; а бяла,
заострена шапка, с дълги ревери отстрани, лежеше върху пуснатата му черна коса.
Това беше облеклото на древното духовенство на Влъхвите\footnote{Зороастрийски
жреци от Персийската империя (``Маги'').}, наричани огнепоклонниците.

``Добре дошли!'' — каза той с тихия си приятен глас, докато един след друг
влизаха в стаята -- ``добре дошъл, Абдус; мир на вас, Родаспе и Тигран, и на теб
татко, Абгар. Всички сте добре дошли, тази къщата се озарява с радостта от
вашето присъствие.''

Бяха деветима мъже, различаващи се значително по възраст, но приличащи си по
богатите одежди от многоцветна коприна и по масивните златни яки около вратовете
си, обозначавайки ги като Партски благородници, и по златни кръгове с крила
почиващи на гърдите им, знакът на последователите на Зороастър.

Те заеха местата си около малък черен олтар в края на стаята, където гореше слаб
пламък. Артабан, застанал до него и размахвайки клонки от тънки
тамариски\footnote{``Barsom'': сноп свещена трева.} над огъня, го хранеше със
сухи борови съчки и благоуханни масла. Тогава той започна древното песнопение на
Ясна и гласовете на неговите спътници се присъединиха към красивите възпявания
на Ахура-Мазда:
\begin{center}
\fontsize{8}{10}\selectfont
\setlength{\leftskip}{1cm}
Почитаме Божественият Дух, \\
цялата мъдрост и доброта притежаващ, \\
Заобиколени от Светите Безсмъртни, \\
дарителите на награда и благодат. \\
Делата на Неговите ръце ни правят щастливи, \\
Неговата истина и Неговата сила споделяме. \\
Възхваляваме всички неща, които са чисти, \\
защото те са Неговото единствено Творение; \\
Мислите, които са истински, и думите \\
и дела, спечелили одобрение; \\
Те са подкрепени от Него, \\
и пред тях се прекланяме. \\
Чуй ни, о, Мазда! Ти живееш \\
в истина и в небесна радост; \\
Очисти ни от лъжата и пази ни \\
от злото и робуването на греха; \\
Излей светлината и радостта от Твоят живот \\
върху нашия мрак и тъга. \\
Освети нашите градини и ниви, \\
Освети нашата работа и лъкатушене; \\
Освети цялата човешка раса, \\
Вярващи и невярващи; \\
Освети ни сега през нощта, \\
Освети ни сега с Твоята мощ, \\
Пламъкът на нашата свята любов \\
и песента на нашето поклонение, приемащ.
\end{center}

Огънят се издигаше с песнопението, пулсиращо, сякаш беше направен от музикален
пламък, докато хвърли ярка светлина през целото помещение, разкривайки неговата
простота и великолепие.

Подът беше положен с тъмносини плочки с бели жилки; пиластри от усукано сребро
се открояваха на фона на сините стени; Клеристория\footnote{Висока част от
стена, с прозорци над нивото на очите (``Clerestory'').} от прозорци със
закръглени сводове над които беше окачена лазурна коприна; сводестият таван беше
настилка от сапфири, като небесата в своята чистота, осеяни със сребърни звезди.
От четирите ъгъла на покрива висяха четири златни магически колела, наричани
езиците на боговете. В източния край, зад олтара, имаше два тъмночервени стълба
от порфир; над тях трегер от същия камък, върху който беше изсечена фигурата на
крилат стрелец, със стрела поставена на тетивата и обтегнат лък.

Вратата между колоните, която се отваряше към терасата на покрива, беше покрита
с тежка завеса с цвят на зрял нар, избродиран с безброй златни лъчи, изстрелващи
се нагоре от пода. Всъщност стаята беше като тиха, звездна нощ, цялата лазурна и
сребриста, сияеща на изток с розовото обещание на зората. Тя беше, както трябва
да бъде къщата на един мъж, израз на характера и духа на господарят.

Когато песента приключи, той се обърна към приятелите си и ги покани да седнат
на дивана в западния край на стаята.

``Дойдохте тази вечер,'' каза той, оглеждайки кръга, ``по моя покана, като верни
ученици на Зороастър, да подновите вашето поклонение и разпалите отново вярата
си в Бога на Чистотата, като този огън който се разгаря на олтара. Ние се
покланяме не пред огъня, а пред Този, от който това е избраният символ, защото е
най-чистото от всички сътворени неща. То ни говори за някой, който е Светлина и
Истина. Не е ли така, татко?''

``Добре казано, сине мой'' — отговори почитаният Абгар. ``Просветените никога не
са идолопоклонници. Те повдигат воала на формата и влизат в светилището на
реалността и нова светлина и истина идват при тях непрекъснато чрез старите
символи.'' ``Чуйте ме тогава, отец мой и приятелите мои'' — каза Артабан много
тихо, ``докато ви разказвам за новата светлина и истина, които дойдоха при мен
чрез най-древните от всички знаци. Заедно търсихме тайните на природата и
изучавахме лечебните свойства на водата, огъня и растенията. Чели сме също и
книгите с пророчества, в които бъдещето е мъгляво предсказано с думи, които са
трудни за разбиране. Но най-висшето от всички учения е познанието за звездите.
Да проследиш пътят им означава да разплетеш нишките на мистерията на живота от
началото до края. Ако можехме да ги следваме безгрешно, нищо нямаше да бъде
скрито от нас. Но не е ли все още непълно познанието ни за тях? Нима няма още
много звезди отвъд нашия хоризонт -- светлини, които са познати само на
обитателите на далечната южна земя, сред дърветата с подправки на Пунт и златните
мини на Офир?''

Сред слушателите се разнесе шепот на съгласие.

``Звездите'', каза Тигран, ``са мислите на Вечния. Те са безброй. Но мислите на
човека могат да се преброят, като годините на живота му. Мъдростта на Влъхвите е
най-великата от всички мъдрости на земята, защото познава собственото си
невежество. И това е тайната на властта. Ние караме хората винаги да търсят и да
чакат нов изгрев. Но ние самите знаем, че тъмнината е равна на светлината и че
конфликтът между тях никога няма да приключи.''

``Това не ме удовлетворява'', отвърна Артабан, ``защото, ако чакането трябва да
продължи вечно, ако то не може да бъде възнаградено, тогава не би било мъдрост
да гледаш и чакаш. Трябва да станем като онези нови учители на гърците, които
казват, че няма истина и че единствените мъдреци са онези, които прекарват
живота си в откриване и разобличаване на лъжите, в които светът е вярвал. Но
новият изгрев със сигурност ще дойде в определеното време. Не ни ли казват
нашите собствени книги, че това ще се случи и че хората ще видят блясъка на
голяма светлина?''

``Вярно е'', каза гласът на Абгар; ``всеки верен ученик на Зороастър познава
пророчеството на Авеста и носи словото в сърцето си. <<В онзи ден Сосиош
Победителят ще се въздигне от многобройните пророци в източната страна. Около
него ще блести могъща светлина и той ще направи живота вечен, нетленен и
безсмъртен, а мъртвите ще възкръснат.>>''

``Това е мрачно пророчество'', каза Тигран, ``и може би никога няма да го
разберем. По-добре е да вземем предвид нещата, които са близо под ръка, и да
увеличим влиянието на Влъхвите в собствената им страна, вместо да търсим някой,
който може да е непознат и на когото трябва да преотстъпим властта си.''

Останалите сякаш одобряваха тези думи. Мълчаливо усещане за съгласие се
проявяваше между тях; погледите им откликваха с този неопределим израз, който
винаги следва, когато говорещият е изказал мисълта, която се е криела в сърцата
на неговите слушатели. Но Артабан се обърна към Абгар със сияние на лицето си и
каза:

``Татко, пазил съм това пророчество в тайното място на душата си. Религия без
голяма надежда би била като олтар без жив огън. И сега пламъкът гори по-ярко, и
под светлината му прочетох непознати вести, които също са доишли от извора на
Истината и говорят дори още по-ясно за издигането на Победоносния в неговия
блясък.''

Той извади от пазвата на туниката си два малки свитъка от фин лен, с послания
върху тях, и ги разгърна внимателно на коляното си.

``В годините, които са изгубени в миналото, много преди да дойдат нашите предци
в земите на Вавилон, е имало мъдреци в Халдея, от които първите от Влъхвите са
научили небесните тайни. И от тези, Валаам синът на Веор е бил един от
най-могъщите. Чуйте думите на неговото пророчество: <<Ще изгрее звезда от Яков и
ще се въздигне скиптър от Израел.>>''

Устните на Тигран се свиха надолу с презрение, когато той каза:

``Юда беше пленник при водите на Вавилон и синовете на Яков бяха в робство на
нашите царе. Племена на Израел са разпръснати по планините като изгубени овце и
от малкото останали да обитават Юдея под игото на Рим, няма да се издигне нито
звезда, нито скиптър.''

\end{document}
